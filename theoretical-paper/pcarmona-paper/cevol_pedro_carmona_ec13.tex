\documentclass[11pt,portuguese,a4paper]{article}
\usepackage{includes}

\title{\LARGE{\textbf{Revisão e Reflexão do Artigo:}}\\ Transposition - A Biologically Inspired Mechanism to Use with Genetic Algorithms \cite{Simoes99transposition:a}}
\author{ \textbf{Pedro Carmona} \\ Computação Evolucionária\\ MEI -2013/2014 \\ Universidade de Coimbra \\ pcarmona@student.dei.uc.pt}
\date{\today}

\renewcommand\lstlistingname{Listagem}
\usepackage[square,sort,comma,numbers]{natbib}

%%% BEGIN DOCUMENT
\begin{document}

\maketitle

\begin{abstract}
Este trabalho procura rever e refletir sobre um artigo que apresenta um mecanismo de recombinação inspirado na biologia celular, com o nome de transposição. Para isso, é exposta a inspiração que levou ao algoritmo, assim como o algoritmo e os resultados. As decisões tomadas são credíveis e com resultados comprovaram a sua eficiência, ainda assim, são propostas questões e alternativas aos autores de \cite{Simoes99transposition:a}.
\end{abstract}

\section{What}

%Aqui se descreve a referência do trabalho e se dá o contexto, isto é, a área e o \textbf{tema} do artigo (do ponto de vista do autor).

O artigo revisto, \cite{Simoes99transposition:a}, tem como área de estudo a programação inspirada na biologia evolucionária.
Nesta área estudaram-se desde cedo e com maior difusão dois operadores genéticos de variação, sendo eles a mutação de genes de um cromossoma e o cruzamento de cromossomas.
%Nestes estudos sobre operadores genéticos de variação foi retirada a conclusão de que a mutação não é suficiente só por si.

Na biologia existem ainda outros métodos de recombinação menos populares, como por exemplo \textit{transformation, transduction, conjugation, retroinsertion, fusion, unequal recombination, transposition}\cite{Simoes99transposition:a}. De estes métodos alternativos, que envolvem mecanismos como a inserção, a duplicação ou movimento, foi selecionado um método pelos autores para a realização deste artigo.

O artigo debruça-se sobre um método diferente de recombinação genética e presente na biologia celular, a transposição.
Na biologia celular, a transposição permite a recombinação de genes através de permutação dentro do cromossoma, ou por vezes também permutações entre cromossomas.
A transposição é um mecanismo que acontece devido a unidades móveis genéticas chamadas transpósons, que podem se manifestar de várias maneiras. Podem movimentar-se para zonas novas do mesmo cromossoma, ou ainda para outros cromossomas, outros transpóson deixam cópias nos locais destino, e existem ainda transpósons que fazem cópias desde a sua posição original e enviam essas copias para outros locais.


\section{Why}

%Porque é que o \textbf{autor} acha a questão relevante? Porque é que \textbf{nós} achamos a questão relevante? Aqui se colocam as respostas a estas duas questões.
Na programação genética, a mutação é um mecanismo que aplicado com uma pequena probabilidade, permite misturar um pouco algumas zonas do cromossoma, permitindo continuar a evolução do algoritmo genético. Pelo contrário, o cruzamento é um mecanismo que permite maior variação dos cromossomas, ao mesmo tempo que produz descendentes com qualidade.  Tomando como ponto de partida a estrutura padrão da programação genética, e sabendo que o operador transposição pode assumir mecanismos semelhantes ao cruzamento, os autores procuraram substituir o cruzamento pela transposição.

\section{Contribution(s)}

%O que \textbf{propõe} o autor? Caso exista uma análise experimental, quais os \textbf{resultados} que suportam as conclusões do autor?
O autor propõe um método para programar a transposição, no qualassume trocas de "material genético" entre dois indivíduos.
São escolhidos dois progenitores, e no primeiro progenitor é procurado o transpóson.
Os autores estabeleceram que é trocada a mesma quantidade de informação entre os dois cromossomas de acordo com o ponto de inserção no cromossoma destino.
O processo para a escolha de um transposen, é realizado num dos progenitores:
\begin{itemize}
    \item É escolhido um gene do progenitor aleatoriamente. Os genes imediatamente antes do gene escolhido ao acaso formam a \textit{flanking sequence lenght} (FSL).
    \item A segunda FSL que corresponde ao final do transposen é procurada a partir do gene escolhido ao acaso, dando a volta ao cromossoma se for necessário.
\end{itemize}
Depois disto é efetuada uma troca de "material genético" com outro progenitor, sendo o ponto de inserção no progenitor identificado pelo complexo FSL.
No entanto, os autores refere a solução para o caso de não ser encontrado nenhum transpósen no primeiro progenitor, no qual não se faz a transposição nesses progenitores, e portanto os filhos são copias dos pais. Refere ainda que para o caso de não ser encontrado o FSL correspondente no segundo progenitor, a inserção do transpósen é realizada num ponto aleatório do segundo progenitor.

Foi realizado um teste para escolher o melhor comprimento de FSL (Fig. 10 in \cite{Simoes99transposition:a}), que permitiu observar que o melhor comprimento FSL seria 4. Foi também possível provar que quanto maior o comprimento do FSL, menor é o performance deste operado, um exemplo disso é no caso com 50 indivíduos de população, o pior dos casos é 10, que é ao mesmo tempo o maior comprimento de FSL deste teste.

Outros testes feitos para comparar o novo mecanismo, a transposição, com o mecanismo de cruzamento, tiveram em conta as variações existentes de cruzamento, nomeadamente o cruzamento uniforme (CZU), o cruzamento num ponto (CZ1P) e o cruzamento em dois pontos (CZ2P).
O novo operador, com 50 elementos de população, teve melhores resultados que todas as variações de CZ1P em 50, 100 e 200, assim como no CZU para as mesmas populações.
No caso do CZ2P, os resultados foram superiores comparando a transposição de 50 elementos de população com o CZ2P de 50 e 100 elementos. Para ser superior ao  CZ2P com 200 de população, a transposição precisou apenas de 100 elementos de população.
Os resultados suportam a conclusão do autor, assim como inspiraram mais estudos no mesmo projeto \cite{Simoes00usinggenetic}.


\section{Like/ Don't Like}

Concordo com introdução deste novo mecanismo, que foi inspirado na biologia celular, e que obteve resultados promissores. Os autores aplicaram a transposição como um mecanismo para recombinação que favorece a procura global.

A meu ver a transposição também se poderia aplicar em mecanismos de recombinação que favorecem a procura local, onde o transpósen identificado era submetido a uma mutação, simulando que o transpósen sai do cromossoma e outro transpósen assumiu o seu lugar. Ainda que fosse utilizado para procura global, este método poderia também ser utilizado para colmatar o caso em que não é encontrado FSL no segundo progenitor, em vez da inserção aleatória. Esta abordagem poderia favorecer a transposição em casos de estudo em que a ordem interessa.
Para o caso de não ser encontrado nenhum transpósen no primeiro progenitor, uma abordagem que sugeria seria fazer essa procura no segundo progenitor.

%Esta é uma parte relevante do trabalho, para efeitos de avaliação. Identifique os aspetos \textbf{positivos} e \textbf{negativos} da proposta. Que \textbf{você} alternativas sugere? As ideias do autor podem ser adaptadas a outras questões?\\

\bibliographystyle{IEEEtran}
\renewcommand{\bibfont}{\small}
\bibliography{cevol_pedro_carmona_ec13}

\end{document}
