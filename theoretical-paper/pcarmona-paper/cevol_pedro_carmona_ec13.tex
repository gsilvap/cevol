\documentclass[12pt,portuguese,a4paper]{article}
\usepackage{includes}

\title{\LARGE{\textbf{Revisão e Reflexão do Artigo:}}\\ Transposition - A Biologically Inspired Mechanism to Use with Genetic Algorithms \cite{Simoes99transposition:a}}
\author{ \textbf{Pedro Carmona} \\ Computação Evolucionária\\ MEI -2013/2014 \\ Universidade de Coimbra \\ pcarmona@student.dei.uc.pt}
\date{\today}

\renewcommand\lstlistingname{Listagem}

%%% BEGIN DOCUMENT
\begin{document}

\maketitle

\begin{abstract}

O resumo é o local onde se faz a síntese do documento.
Este texto pretende ilustrar o que se pretende com a \textbf{ficha de leitura} relativa ao texto teórico que contam para a avaliação na cadeira. São descritas as suas várias secções e o que se pretende com cada uma delas.  Este documento deve ter um \textbf{máximo de três páginas de texto}, não sendo obrigatório a sua escrita em \LaTeX{}. A inclusão de figuras e/ou tabelas não  interfere com este limite.Pode ser escrito em português ou em inglês.
\end{abstract}

\section{What}

%Aqui se descreve a referência do trabalho e se dá o contexto, isto é, a área e o \textbf{tema} do artigo (do ponto de vista do autor).

O artigo revisto, \cite{Simoes99transposition:a}, tem como área de estudo a programação inspirada na biologia evolucionária.
Nesta área estudaram-se desde cedo e com maior difusão dois operadores genéticos de variação, sendo eles a mutação de genes e o cruzamento de cromossomas. Nestes estudos sobre operadores genéticos de variação foi retirada a conclusão de que a mutação não é suficiente só por si.

Na biologia existem ainda outros métodos de recombinação menos populares, como por exemplo \textit{transformation, transduction, conjugation, retroinsertion, fusion, unequal recombination, transposition}\cite{Simoes99transposition:a}. De estes métodos alternativos, que envolvem mecanismos como a inserção, a duplicação ou movimento, foi selecionado um método pelos autores para a realização deste artigo.

O artigo debruça-se sobre um método diferente de recombinação genética e presente na biologia celular, a transposição.
Na biologia celular, a transposição permite a recombinação de genes através de permutação dentro do cromossoma, ou por vezes também permutações entre cromossomas.
A transposição é um mecanismo que acontece devido a unidades móveis genéticas chamadas \textit{transposons}.
Os movimentos dos \textit{transposons} podem se manifestar de várias maneiras:
\begin{itemize}
    \item Alguns \textit{transposons} movimentam-se para zonas novas do mesmo cromossoma, ou ainda para outros cromossomas;
    \item Outros \textit{transposons} deixam cópias nos locais destino;
    \item Existem ainda \textit{transpóson} que fazem cópias desde a sua posição original e enviam essas copias para outros locais.
\end{itemize}

\paragraph{Transposição Computacional}

Na programação genética, a mutação é um mecanismo que aplicado com uma pequena probabilidade, permite misturar um pouco algumas zonas do cromossoma, permitindo continuar a evolução do algoritmo genético. Pelo contrário, o cruzamento é um mecanismo que permite maior variação dos cromossomas, ao mesmo tempo que produz descendentes com qualidade.  Tomando como ponto de partida a estrutura padrão da programação genética, e sabendo que o operador transposição pode assumir mecanismos semelhantes ao cruzamento, foi tomada a decisão de substituir o cruzamento pela transposição.

Depois de serem escolhidos dois progenitores, é procurado

No trabalho \cite{Simoes99transposition:a} são ainda expostos resultados promissores, mas ao mesmo tempo os autores consideram ser um estudo ainda curto para tirar conclusões finais.



Computational Transposition
After selecting two parents for mating we look for the transposon in one of them.
The same amount of genetic material is exchanged between the two chromosomes according to the found insertion point
Now, we are going to describe how the transposon will be formed, how it will move in the genome, how to define the insertion point, how to define the flanking sequences length and how the integration in the new position will take place.

-Choose at random a gene
-flanking sequence lenght (FSL) genes immediately before gene T
-second flanking sequence can be identical or inverse to the first one

- In case no other FSL found there will be no transposition.
- If there is no equal or inverse sequence in the target chromosome, the insertion point is defined randomly


\section{Why}

%Porque é que o \textbf{autor} acha a questão relevante? Porque é que \textbf{nós} achamos a questão relevante? Aqui se colocam as respostas a estas duas questões.

\section{Contribution(s)}

%O que \textbf{propõe} o autor? Caso exista uma análise experimental, quais os \textbf{resultados} que suportam as conclusões do autor?


\section{Like/ Don't Like}

%Esta é uma parte relevante do trabalho, para efeitos de avaliação. Identifique os aspetos \textbf{positivos} e \textbf{negativos} da proposta. Que \textbf{você} alternativas sugere? As ideias do autor podem ser adaptadas a outras questões?\\

\bibliographystyle{IEEEtran}
\bibliography{cevol_pedro_carmona_ec13}

\end{document}
