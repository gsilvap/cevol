\documentclass[12pt,portuguese,a4paper]{article}
\usepackage{includes}

\title{\Large{\textbf{Título}}}
\author{ Pedro Carmona \\ Computação Evolucionária\\ MEI -2013/2014 \\ Universidade de Coimbra \\ -- email --}
\date{\today}

\renewcommand\lstlistingname{Listagem}


%%% BEGIN DOCUMENT
\begin{document}




\maketitle


\begin{abstract}

O resumo é o local onde se faz a síntese do documento.
Este texto pretende ilustrar o que se pretende com a \textbf{ficha de leitura} relativa ao texto teórico que contam para a avaliação na cadeira. São descritas as suas várias secções e o que se pretende com cada uma delas.  Este documento deve ter um \textbf{máximo de três páginas de texto}, não sendo obrigatório a sua escrita em \LaTeX{}. A inclusão de figuras e/ou tabelas não  interfere com este limite.Pode ser escrito em português ou em inglês.
\end{abstract}

\section{What}

Aqui se descreve a referência do trabalho e se dá o contexto, isto é, a área e o \textbf{tema} do artigo (do ponto de vista do autor).

\section{Why}

Porque é que o \textbf{autor} acha a questão relevante? Porque é que \textbf{nós} achamos a questão relevante? Aqui se colocam as respostas a estas duas questões.

\section{Contribution(s)}

O que \textbf{propõe} o autor? Caso exista uma análise experimental, quais os \textbf{resultados} que suportam as conclusões do autor?


\section{Like/ Don't Like}

Esta é uma parte relevante do trabalho, para efeitos de avaliação. Identifique os aspetos \textbf{positivos} e \textbf{negativos} da proposta. Que \textbf{você} alternativas sugere? As ideias do autor podem ser adaptadas a outras questões?\\

\noindent {\Large \textbf{Bibliografia}}\\

Coloque nesta parte final a bibliografia adicional que eventualmente leu/ajudou a perceber a alisar o trabalho.




\bibliographystyle{plain}
% retirar coment‡rio na pr—xima linha para incluir bibliografia
%\bibliography{??}

\end{document}
