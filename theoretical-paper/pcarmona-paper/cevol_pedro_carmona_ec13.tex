\documentclass[12pt,portuguese,a4paper]{article}
\usepackage{includes}

\title{\LARGE{\textbf{Revisão e Reflexão do Artigo:}}\\ Transposition - A Biologically Inspired Mechanism to Use with Genetic Algorithms \cite{Simoes99transposition:a}}
\author{ \textbf{Pedro Carmona} \\ Computação Evolucionária\\ MEI -2013/2014 \\ Universidade de Coimbra \\ pcarmona@student.dei.uc.pt}
\date{\today}

\renewcommand\lstlistingname{Listagem}

%%% BEGIN DOCUMENT
\begin{document}

\maketitle

\begin{abstract}
Este trabalho procura rever e refletir sobre um artigo que apresenta um mecanismo de recombinação inspirado na biologia celular, com o nome de transposição.
\end{abstract}

\section{What}

%Aqui se descreve a referência do trabalho e se dá o contexto, isto é, a área e o \textbf{tema} do artigo (do ponto de vista do autor).

O artigo revisto, \cite{Simoes99transposition:a}, tem como área de estudo a programação inspirada na biologia evolucionária.
Nesta área estudaram-se desde cedo e com maior difusão dois operadores genéticos de variação, sendo eles a mutação de genes de um cromossoma e o cruzamento de cromossomas.
%Nestes estudos sobre operadores genéticos de variação foi retirada a conclusão de que a mutação não é suficiente só por si.

Na biologia existem ainda outros métodos de recombinação menos populares, como por exemplo \textit{transformation, transduction, conjugation, retroinsertion, fusion, unequal recombination, transposition}\cite{Simoes99transposition:a}. De estes métodos alternativos, que envolvem mecanismos como a inserção, a duplicação ou movimento, foi selecionado um método pelos autores para a realização deste artigo.

O artigo debruça-se sobre um método diferente de recombinação genética e presente na biologia celular, a transposição.
Na biologia celular, a transposição permite a recombinação de genes através de permutação dentro do cromossoma, ou por vezes também permutações entre cromossomas.
A transposição é um mecanismo que acontece devido a unidades móveis genéticas chamadas transpósons, que podem se manifestar de várias maneiras. Podem movimentar-se para zonas novas do mesmo cromossoma, ou ainda para outros cromossomas, outros transpóson deixam cópias nos locais destino, e existem ainda transpósons que fazem cópias desde a sua posição original e enviam essas copias para outros locais.


\section{Why}

%Porque é que o \textbf{autor} acha a questão relevante? Porque é que \textbf{nós} achamos a questão relevante? Aqui se colocam as respostas a estas duas questões.
Na programação genética, a mutação é um mecanismo que aplicado com uma pequena probabilidade, permite misturar um pouco algumas zonas do cromossoma, permitindo continuar a evolução do algoritmo genético. Pelo contrário, o cruzamento é um mecanismo que permite maior variação dos cromossomas, ao mesmo tempo que produz descendentes com qualidade.  Tomando como ponto de partida a estrutura padrão da programação genética, e sabendo que o operador transposição pode assumir mecanismos semelhantes ao cruzamento, os autores procuraram substituir o cruzamento pela transposição.

\section{Contribution(s)}

%O que \textbf{propõe} o autor? Caso exista uma análise experimental, quais os \textbf{resultados} que suportam as conclusões do autor?
O método utilizado para programar a transposição assume trocas de "material genético" entre dois indivíduos.
Depois de serem escolhidos dois progenitores, é procurado um transpóson em um dos pais.
Os autores estabeleceram que é trocada a mesma quantidade de informação entre os dois cromossomas de acordo com o ponto de inserção no cromossoma destino.
O processo para a escolha de um transposen, é realizado num dos progenitores:
\begin{itemize}
    \item É escolhido um gene do progenitor aleatoriamente. Os genes imediatamente antes do gene escolhido ao acaso formam a \textit{flanking sequence lenght} (FSL).
    \item A segunda FSL que corresponde ao final do transposen é procurada a partir do gene escolhido ao acaso, dando a volta ao cromossoma se for necessário.
\end{itemize}
Depois disto é efetuada uma troca de "material genético" com outro progenitor, sendo o ponto de inserção no progenitor identificado pelo complexo FSL.
No entanto, os autores referem resoluções para o caso de não ser encontrado nenhum transpósen no primeiro progenitor, no qual não se faz a transposição nesses progenitores, e portanto os filhos são copias dos pais. Refere ainda que para o caso de não ser encontrado o FSL correspondente no segundo progenitor, a inserção do transpósen é realizada num ponto aleatório do segundo progenitor.

\section{Like/ Don't Like}

%Esta é uma parte relevante do trabalho, para efeitos de avaliação. Identifique os aspetos \textbf{positivos} e \textbf{negativos} da proposta. Que \textbf{você} alternativas sugere? As ideias do autor podem ser adaptadas a outras questões?\\

\bibliographystyle{IEEEtran}
\bibliography{cevol_pedro_carmona_ec13}

\end{document}
